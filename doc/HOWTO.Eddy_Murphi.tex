\documentclass{article}

\begin{document}

\section{Installing}

Just compile the Murphi compiler:

{\tt cd src; make clean; make}

The final executable name is {\tt mu}.

\section{Using the examples}

Predefined examples are in the directory ex. In every subdirectory of ex/
(except for scripts) there is at least one protocol example (extension .m).

\subsection{Using Makefiles}

In every subdirectory of ex/ (except for scripts) there are also 6 makefiles:

\begin{description}

	\item[Makefile.FULL] creates model.C from model.m, then compiles model.C
in order to obtain the executable model, i.e. the standard Murphi verifier.
Without parameters, compiles all the (predefined) model.m in the directory.
Possible parameters: a single model, or "clean" (deletes .C and executables
files). There are two Makefile constants which are left for user specification:

	\begin{description}

		\item[{\tt MURPHIOPT}] to specify extra options to the Murphi
compiler (when generating model.C)

		\item[{\tt CFLAGSOPT}] to specify extra compiler options when
compiling model.C

	\end{description}

	\item[Makefile.FULL.bc] the same as Makefile.FULL, but enables hash
compaction (only if model.C does not exist or is out-of-date w.r.t. model.m)

	\item[Makefile.MPI] the same as Makefile.FULL, but creates an executable
of Eddy\_Murphi (only if model.C does not exist or is out-of-date w.r.t.
model.m).

	\item[Makefile.MPI.bc] the same as Makefile.MPI, but enables hash
compaction (see Makefile.FULL.bc)

	\item[Makefile.MPI.bc.old] the same as Makefile.MPI.bc, but creates an
executable of an MPI porting of Stern's PMurphi (hash compaction is required).

	\item[Makefile] link to Makefile.MPI.bc

\end{description}

Thus, for example, to make the model adash in original Murphi with hash
compaction (from ex/dash directory), just type:

{\tt make -f Makefile.FULL.bc adash; ./adash -m$m$ -ndl}

On the other hand, to compile the same model with Eddy\_Murphi (hash compaction
enabled), just type:

{\tt make adash}

Note that {\tt make -f Makefile.FULL adash MURPHIOPT=``-b -c''} would have
produced the same result.

%Note that {\tt make -f Makefile.MPI.bc adash} would have produced the same
%result, since Makefile points to Makefile.MPI.bc.

\subsection{Running verifications}

To run a verification with the original Murphi, the following options are
useful:

\begin{description}

	\item[-m$n$] where $n$ is the amount of RAM available for the
verification (in MB)

	\item[-ndl] disables deadlock detection

	\item[-p$n$] prints verification progress reports every $10^{n}$ states.

\end{description}

To run a verification with Eddy\_Murphi, the same options as above are still
available (the -m option of course is meant to give the amount of memory on each
process), plus these two:

\begin{description}

	\item[-bsv$n$] is the value for BUFSIZE, i.e. there will be $n$ states
in each queue line (default is 1024)

	\item[-bcv$n$] is the value for BUFCOUNT, i.e. there will be $n$ queue
lines (default is 8)

\end{description}

This is a typical run\footnote{\label{lamboot.note}Note however that some MPI
implementations do not require {\tt lamboot} and {\tt lamhalt}, and require more
options to be specified to {\tt mpirun}.} for Eddy\_Murphi (from ex/dash
directory):

{\tt lamboot; mpirun -np $n$ adash -m$m$ -ndl; lamhalt}

where $m$ is the amount of RAM available for the verification for each process
(in MB) and $n$ is the number of processes on which to run the verification. In
this case, BUFSIZE is 1024 and BUFCOUNT is 8.

Note however that error tracing is still not completely tested, and that, in
order to enable it, it is necessary to compile defining the {\tt HASHC\_TRACE}
constant (e.g. {\tt make -f Makefile.MPI.bc adash
CFLAGSOPT=``-DHASHC\_TRACE''}). 

Note moreover that Eddy\_Murphi typically needs to generate huge files during
the verification (especially if error tracing is required). To let user to put
these files in different directories thatn the current one, the following
options are available:

\begin{description}

	\item[-dq $path$] sets where to put the files for the BF queue

	\item[-d $path$] sets where to put the files for the error tracing

\end{description}

Note that on some Linux systems, it is not allowed to create files larger than
2GB. If you're using one of these systems, define the {\tt SPLITFILE} constant
when compiling (e.g. {\tt make -f Makefile.MPI.bc adash
CFLAGSOPT=``-DHASHC\_TRACE -DSPLITFILE''}).

\subsection{Using PBS}

In order to run a verification with Eddy\_Murphi, also a PBS script is available
- for hosts on which PBS is installed\footnote{This script only works for those
systems on which the note \ref{lamboot.note} does not hold.}. The script is in
the scripts/ directory, and its name is create\_PBS.script. It can be executed
in this way (from ex directory):

{\tt ./scripts/create\_PBS.script -f $<$protocol$>$ -N $n$ [-ppn $p$] [-vo
$\backslash$($<$opts$>$$\backslash$)] [-mtf $<$max times file$>$] [-keep] [-m
$<$mem per node$>$] [-nompi] [-noexec]}

where: 

\begin{itemize}

	\item $<$protocol$>$ is the name of the executable for the Eddy\_Murphi
verification (with absolute or relative path; if the file doesn't exists, the
script tries to make it up using Makefile.MPI.bc in the same directory);

	\item $<$n$>$ is the number of nodes on which to run the verification;

	\item $<$p$>$ is the number of processes for each node (default is 1);

	\item $<$opts$>$ are the verification options for $<$protocol$>$ (note
that $\backslash$( and $\backslash$) are mandatory, default is no option);

	\item $<$queue name$>$ is the name of the queue in the PBS (default is
the first enabled queue)

	\item $<$max time$>$ is the maximum verification time allowed (defualt
is the maximum walltime defined for the current queue)

	\item $<$max times file$>$ is a text file where each line is of type 
$<$from\_nodes$>$ $<$to\_nodes$>$ $<$max time$>$, meaning that using a number of
nodes between  $<$from\_nodes$>$ and $<$to\_nodes$>$ the maximum allowed time is
$<$max time$>$ (useful when usage policies are defined)

	\item the option -keep, if given, prevents the deleting of the qsub
script.

	\item $<$mem per node$>$ is the amount of memory (in MB) that PBS has to
reserve on each node

	\item the option -nompi, if given, allows to run the verification
without using mpirun (thus standalone programs not using MPI)

	\item if the option -noexec is given then the PBS script is generated
but not executed.

\end{itemize}

Thus, for example (from ex directory),

{\tt ./scripts/create\_PBS.script -f dash/ldash -N 7 -ppn 1 -vo $\backslash$(-p4
-ndl -m100$\backslash$)}

will post to PBS the request of the verification for the protocol ldash with
options -p4 -ndl -m100, using 7 nodes (1 process per node). If dash/ldash does
not exist, the script will create it by using dash/Makefile.MPI.bc.

\end{document}
